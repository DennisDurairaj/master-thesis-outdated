\documentclass[paper.tex]{subfiles}
\begin{document}
\chapter{Creating PDF documents in front-end}

\section{Frontend technologies}
\par
Front-end web development (also known as client-side development) is the practice of producing websites or Web Applications using HTML, CSS and JavaScript so that an end user can see and interact with them. The challenges that come with front end development is that the tools and techniques used to create the front end of a website change constantly and so the developer needs to constantly be aware of how the field is developing.

\subsection{HTML5, CSS and JavaScript}
\begin{enumerate}[a.]
\item \textbf{What is HTML5?}
  \par
    HTML, or HyperText Markup Language, is the most important element of the World Wide Web. It’s the language used to describe what a webpage should look like. However, HTML on its own is pretty boring because it can only deliver static pages; in order to meet the growing demand for more impressive web features, HTML has been coupled with plugins like CSS, Flash, Java, Silverlight, etc.
    \break
    It has become something of a bloated mess and different browsers implement those features in their own ways. HTML5 is meant to solve HTML’s big problems for a cleaner and more efficient web.
\item \textbf{Cascading Stylesheets}
  \par
    CSS is a style language that defines layout of HTML documents. For example, CSS covers fonts, colours, margins, lines, height, width, background images, advanced positions and many other things. Just wait and see!
    HTML can be (mis-)used to add layout to websites. But CSS offers more options and is more accurate and sophisticated. CSS is supported by all browsers today.
    \newline
    \break
    How do HTML and CSS work together? In general, you use HTML to describe the content of the document, not its style. You use CSS to specify the document's style, not its content.
\item \textbf{JavaScript - The language of the Web}
  \par
    JavaScript is a very powerful client-side scripting language. JavaScript is used mainly for enhancing the interaction of a user with the webpage. In other words, you can make your webpage more lively and interactive, with the help of JavaScript.
    From the browser to the server, JavaScript is one of the most versatile and popular languages powering the modern web.
    \newline
    \break
    Client-side JavaScript extends the core language by supplying objects to control a browser and its Document Object Model (DOM). For example, client-side extensions allow an application to place elements on an HTML form and respond to user events such as mouse clicks, form input, and page navigation.
    \newline
    \break
    Server-side JavaScript extends the core language by supplying objects relevant to running JavaScript on a server. For example, server-side extensions allow an application to communicate with a database, provide continuity of information from one invocation to another of the application, or perform file manipulations on a server.
\end{enumerate}


\end{document}
