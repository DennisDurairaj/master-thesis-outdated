\documentclass[paper.tex]{subfiles}
\begin{document}
\chapter{Creating PDF documents in front-end}

\section{Frontend technologies}
\par
Front-end web development (also known as client-side development) is the practice of producing websites or Web Applications using HTML, CSS and JavaScript so that an end user can see and interact with them. The challenges that come with front end development is that the tools and techniques used to create the front end of a website change constantly and so the developer needs to constantly be aware of how the field is developing.

\section{HTML5, CSS and JavaScript}
\begin{enumerate}[a.]
\item \textbf{What is HTML5?}
  \par
    HTML, or HyperText Markup Language, is the core element of the Internet. It’s the language used to describe how a webpage should be structured. However, HTML on its own is bland because it can only display static pages; so in order to meet the increasing demand for more impressive web features, HTML with plugins like CSS, Flash, Java, Silverlight, etc. create the modern day Internet that we utilise in our daily life.

    However, it has become something of a mess since different browsers have implemented these features in their own ways. With the advent of HTML5, it is meant to solve HTML’s big problems for a cleaner and more efficient web.
\item \textbf{Cascading Stylesheets}
  \par
    CSS is a styling language that defines layout and design of HTML documents. For example, CSS covers margins, lines, width, background images, fonts, colours, height, advanced positioning and many other things.
    HTML can be used (or misused) to add layout to webpages. However, CSS offers additional options and is more accurate and sophisticated. CSS is completely supported by all browsers today.

    In order to use HTML and CSS together, you use HTML to describe the body of the document and CSS to specify the document's layout, visual appearance, style,etc. not its content.
\item \textbf{JavaScript - The language of the Web}
  \par
    JavaScript is a powerful client-side scripting language which has recently found its way into server-side scripting as well thanks to node.js which is a JavaScript based server-side programming framework. JavaScript is used mainly for improving the interaction of a client with the webpage. In other words, you can make your webpage more interactive and user-friendly, with the help of JavaScript.
    Today, from the browser to the server, JavaScript proves to be one of the most popular and versatile languages powering the modern web.

    Client-side JavaScript makes use of the core language by providing special objects to control a browser using its Document Object Model (DOM). For instance, client-side extensions allow front-end developers to respond to user events such as clicks, scrolls, page navigation etc.

    JavaScript has made its way to the server-side fairly recently. It extends the core language by supplying objects used to run JavaScript on a server. For instance, extensions on the server-side allow an application to provide passing of information from one invocation to another of the application, communicate with a database, or perform file manipulations on a server.
\end{enumerate}

\section{Built-in browser PDF converter}
\par
Modern browsers have the capability of saving webpages as a PDF file. This can come in hand in situations where one would like to save certain details. However, this is a read-only method of creating PDF files from webpages and involves no special application or code to achieve this feature. This method fails to save form details which requires more complex procedures since we need to handle validation to produce a valid form/document for the user.

\end{document}
